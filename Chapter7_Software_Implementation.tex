\chapter{Software Implementation}
\label{ch:software_implementation}

\noindent A central contribution of this research is an open-source, reproducible software framework for American option pricing. The full implementation, including the nine algorithms used throughout the thesis, a comprehensive payoff library, and experimental infrastructure, is publicly available at:
\begin{center}
\url{https://github.com/Daniel-Souza7/thesis-new-files}
\end{center}
The framework emphasizes modularity, extensibility and efficiency enabling systematic experimentation. Furthermore, a manual containing instructions for the utilization of the software is provided in Appendix~\ref{att:setup_guide}.

\section{Software Framework Overview}
\label{sec:framework_overview}

The \texttt{optimal\_stopping} Python package implements a production-grade computational environment designed for large-scale benchmarking of American option pricing algorithms. The architecture separates concerns into five core modules: algorithms, data models, payoffs, optimization infrastructure, and execution orchestration. This modular design enables researchers to extend any component independently without modifying the existing codebase, a critical feature for reproducible research.

\subsection{Algorithmic Library}
\label{subsec:algo_library}

Nine distinct algorithmic families are implemented, spanning three paradigms. \textit{Randomized neural networks} (RLSM, SRLSM, RFQI, SRFQI) employ frozen random weights and ridge regression for continuation value estimation, achieving training times measured in seconds. \textit{Trainable neural networks} (NLSM, DOS) optimize all parameters via gradient descent, providing greater expressiveness at the cost of longer training cycles. \textit{Classical methods} (LSM, FQI, EOP) use polynomial or Laguerre basis functions, serving as established benchmarks. Each algorithm inherits from a common base class ensuring consistent interfaces for training, evaluation, and serialization, which enables fair comparison across methods.

\subsection{Comprehensive Payoff Library}
\label{subsec:payoff_library_impl}

As documented in Section~\ref{sec:payoffs}, the framework provides 360 distinct option structures through combinatorial construction: 30 base payoff functionals paired with 12 barrier conditions. These span vanilla options, basket derivatives with various aggregation schemes (arithmetic, geometric, rank-based), path-dependent contracts (Asian, lookback, range), and dispersion products. An automated registration system allows new payoffs to be defined through simple class inheritance, with no manual modifications to lookup tables required. The complete mathematical specification is available in the repository file \texttt{payoffs\_index.tex}.

\subsection{Stochastic Model Suite}
\label{subsec:model_suite}

Seven stochastic models generate underlying asset paths: Geometric Brownian Motion (baseline validation), Heston stochastic volatility (volatility smile calibration), Fractional Brownian Motion (long-range dependence), Rough Heston (volatility roughness), and empirical models based on Yahoo Finance data or user-provided returns via stationary block bootstrap. The path storage system enables pre-generation and reuse of expensive simulations through compressed HDF5 archives, eliminating redundant computation across experiments. This storage mechanism proved essential for the convergence studies in Section~\ref{subsec:convergence_justification}, where millions of paths were shared across multiple algorithm configurations.

\subsection{Analysis and Visualization Tools}
\label{subsec:analysis_tools}

The framework includes specialized utilities for post-processing and presentation. The convergence analysis module generates plots with confidence intervals constructed using $t$-distribution quantiles for small samples, as discussed in Section~\ref{subsec:convergence_analysis}. Exercise boundary visualization tools render stopping regions in state-time space, revealing how algorithmic choices influence optimal policies. Results aggregation modules consolidate hundreds of experimental runs into Excel spreadsheets with summary statistics, formatted execution times, and configuration metadata. Video generation capabilities produce animated visualizations of convergence dynamics, with frames used to construct the static figures presented throughout Chapter~\ref{ch:results}. All visualization tools employ consistent styling and support both publication-quality PDF output and interactive displays.

\subsection{Configuration-Driven Experimentation}
\label{subsec:config_system}

A comprehensive configuration system (\texttt{configs.py}) specifies all experimental parameters declaratively: algorithms to benchmark, payoffs to price, stochastic models, Monte Carlo budgets, hyperparameters, precision levels, and parallelization settings. Parameter sweeps are enabled through Python iterables, allowing systematic convergence studies and sensitivity analyses without code modification. Every numerical result can be traced to its originating configuration snapshot, ensuring full reproducibility. This design eliminates hard-coded values and enables exact replication of all results presented in this thesis by executing the corresponding configuration file against the specified git commit.

\subsection{Hyperparameter Optimization Infrastructure}
\label{subsec:hyperopt_infrastructure}

While the dimension-adaptive heuristics in Equation~\eqref{eq:neuron_heuristic} provide robust default configurations, the framework includes Bayesian optimization capabilities via Optuna \cite{akiba2019optuna} for automatic hyperparameter tuning. The system searches algorithm-specific parameter spaces (hidden layer size, activation function, dropout probability, ridge regularization) using Tree-structured Parzen Estimators with multi-fidelity evaluation to accelerate trials. Complete trial metadata, visualizations, and best configurations are logged automatically. This functionality remains under active development as part of ongoing research into the relationship between problem characteristics and optimal hyperparameter settings, representing a promising direction for future work beyond this thesis.

\section{Interactive Optimal Stopping Game}
\label{sec:pricing_game}

To demonstrate the practical implications of optimal stopping theory and provide an educational tool accessible beyond the academic community, an interactive web application was developed that gamifies the American option exercise decision problem. The game challenges users to compete against pre-trained SRLSM algorithms in real-time stopping decisions, making the abstract mathematics of Chapters~\ref{ch:methods} and \ref{ch:dynamics} tangible through direct interaction.

\subsection{Game Mechanics and Design}
\label{subsec:game_mechanics}

Players observe simulated stock price paths unfolding step-by-step on a visual interface and must decide at each time whether to exercise immediately (locking in the current intrinsic payoff) or continue (preserving optionality for potentially better future states). The machine opponent applies its learned optimal policy in parallel using the trained SRLSM model. Final payoffs are compared to determine the winner, with the human player incentivized to maximize their own payoff while outperforming the algorithmic benchmark.

\vspace{0.5cm}

\noindent Two game modes are available, each highlighting different option pricing complexities:

\begin{enumerate}
    \item \textbf{Up-and-Out Min Put (3 stocks):} A rainbow option where the payoff depends on $\min(S^1_t, S^2_t, S^3_t)$ with an upper barrier at 110. This scenario demonstrates multi-asset correlation effects and knockout risk, requiring players to balance immediate exercise value against both time decay and barrier proximity.

    \item \textbf{Double Knock-Out Lookback Put (1 stock):} A path-dependent contract with payoff $\max_{\tau \le t} S_\tau - S_T$ and barriers at 90 and 110. Players must track the running maximum while navigating a two-sided knockout corridor, illustrating the memory-dependent nature of path-dependent derivatives.
\end{enumerate}

\noindent Both scenarios employ Geometric Brownian Motion with drift $r = 0.02$, volatility $\sigma = 0.2$, and initial spot $S_0 = 100$, consistent with the benchmark parameters used throughout Chapter~\ref{ch:results}. Strike prices are set at $K = 100$ (at-the-money initialization), creating balanced risk-reward profiles where optimal exercise timing is non-trivial.

\subsection{Technical Implementation}
\label{subsec:game_implementation}

The application consists of a React-based frontend providing the graphical interface and a Flask backend serving pre-generated paths and model predictions via a REST API. To eliminate latency from real-time model training during gameplay, the SRLSM algorithms are trained offline on 50,000 Monte Carlo paths for each game mode. Additionally, 500 pre-generated test paths are stored for instant game initialization, ensuring smooth user experience without computational delays.

\vspace{0.5cm}

\noindent The user interface adopts a retro arcade aesthetic with pixel fonts, neon color schemes, and CRT screen effects to enhance engagement and distinguish the educational tool from traditional financial software. Stock prices animate with smooth interpolation between time steps, and exercise decisions are highlighted with visual feedback indicating whether the choice aligns with or deviates from the machine's policy. Real-time statistics track cumulative payoffs, decision agreement rates, and performance rankings.

\vspace{0.5cm}

\noindent Figures~\ref{fig:game_menu} and \ref{fig:game_play} illustrate the interface. The main menu (Figure~\ref{fig:game_menu}) presents scenario selection with brief descriptions of payoff structures and barrier conditions. During gameplay (Figure~\ref{fig:game_play}), the interface displays synchronized stock price charts, decision prompts with hold/exercise buttons, barrier level indicators, and a comparison panel showing both human and machine cumulative payoffs.

\begin{figure}[h]
\centering
% \includegraphics[width=0.7\textwidth]{Chapter7/figures/game_menu.pdf}
\caption{Main menu of the interactive optimal stopping game, allowing players to select between Up-and-Out Min Put and Double Knock-Out Lookback Put scenarios. Each option displays contract specifications and difficulty indicators.}
\label{fig:game_menu}
\end{figure}

\begin{figure}[h]
\centering
% \includegraphics[width=0.85\textwidth]{Chapter7/figures/game_play.pdf}
\caption{Gameplay interface during an active session. The top panel shows real-time stock price evolution with barrier levels marked as horizontal lines. The bottom panel presents exercise decision prompts and tracks cumulative payoffs for both human and machine players, with color-coded indicators for decision agreement.}
\label{fig:game_play}
\end{figure}

\subsection{Educational Value and Public Engagement}
\label{subsec:game_value}

The game serves dual purposes beyond traditional academic dissemination. First, it provides an intuitive demonstration of the research output's practical utility: trained SRLSM models, which appear as abstract mathematical objects in Chapter~\ref{ch:methods}, become tangible decision-making agents that users can directly challenge. This tangibility makes the contribution accessible to finance practitioners, students, and general audiences unfamiliar with stochastic calculus or dynamic programming.

\vspace{0.5cm}

\noindent Second, the game functions as a pedagogical tool for teaching option pricing theory. By experiencing the consequences of suboptimal exercise decisions through repeated gameplay, users develop intuition for concepts such as time value erosion, volatility impact, and early exercise premiums that are difficult to convey through equations alone. Preliminary classroom deployment suggests that students who interact with the game before studying American option theory demonstrate improved conceptual understanding and retention compared to traditional lecture-only instruction.

\vspace{0.5cm}

\noindent The complete source code, including deployment configurations for cloud hosting via Vercel, is available in the \texttt{thesis-game/} subdirectory of the repository. Unlike the core pricing algorithms which reside in the main \texttt{optimal\_stopping/} package, the game is maintained as a standalone application to facilitate independent deployment and distribution. Setup instructions, API documentation, and example deployment scripts are provided in the game's dedicated README file.

\section{Accessibility and Reproducibility}
\label{sec:accessibility}

All software components prioritize transparency and ease of replication. The GitHub repository includes comprehensive documentation with installation procedures, usage examples, and troubleshooting guides. Every experiment presented in Chapter~\ref{ch:results} is tagged with configuration metadata and git commit hashes, enabling exact reproduction from the corresponding codebase snapshot. Pre-generated path datasets used for benchmarking are publicly hosted on Google Drive, eliminating the computational barrier for researchers lacking access to high-performance computing resources.

\vspace{0.5cm}

\noindent For step-by-step instructions on environment setup, dependency installation, executing pricing experiments, generating convergence plots, and extending the codebase with custom algorithms or payoffs, readers are directed to Appendix~\ref{att:setup_guide}, which provides a comprehensive quickstart guide designed for users with basic Python proficiency but no prior familiarity with the framework. Additional documentation for the hyperparameter optimization module and path storage system is available in module-specific README files within the repository.

\vspace{1cm}

\noindent By open-sourcing a production-grade implementation with extensive documentation and reproducible workflows, this research lowers the barrier to entry for future investigations into randomized neural network methods for American option pricing. The modular architecture facilitates extensions to new algorithms, exotic derivatives, and alternative stochastic models, positioning the codebase as a platform for continued methodological development within the computational finance community.
