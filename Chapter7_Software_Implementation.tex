\chapter{Software Implementation}
\label{ch:software_implementation}

\noindent A central contribution of this research is an open-source, reproducible software framework for American option pricing. The full implementation, including the nine algorithms used throughout the thesis, a comprehensive payoff library, and experimental infrastructure, is publicly available at:
\begin{center}
\url{https://github.com/Daniel-Souza7/thesis-new-files}
\end{center}
The framework emphasizes modularity, extensibility and efficiency enabling systematic experimentation. Furthermore, a manual containing instructions for the utilization of the software is provided in Appendix~\ref{att:setup_guide}.

\section{Software Framework Overview}
\label{sec:framework_overview}

The \texttt{optimal\_stopping} Python package implements a production-grade computational environment for large-scale American option pricing benchmarks. Nine algorithmic families span three paradigms: \textit{randomized neural networks} (RLSM, SRLSM, RFQI, SRFQI) using frozen weights and ridge regression with training times in seconds, \textit{trainable neural networks} (NLSM, DOS) optimizing via gradient descent, and \textit{classical methods} (LSM, FQI, EOP) using polynomial or Laguerre basis functions. The 360-option library documented in Section~\ref{sec:payoffs} combines 30 base payoff functionals with 12 barrier conditions through automated registration enabling simple class inheritance. Seven stochastic models generate paths: Geometric Brownian Motion, Heston stochastic volatility, Fractional Brownian Motion, Rough Heston, and empirical models via Yahoo Finance or custom data bootstrap. The path storage system uses compressed HDF5 archives enabling pre-generation and reuse, eliminating redundant computation as demonstrated in Section~\ref{subsec:convergence_justification}.

\vspace{0.5cm}

\noindent Specialized utilities handle post-processing: convergence analysis with $t$-distribution confidence intervals for small samples, exercise boundary visualization in state-time space, Excel aggregation with summary statistics, and video generation producing the static figures in Chapter~\ref{ch:results}. A comprehensive configuration system (\texttt{configs.py}) specifies all parameters declaratively with parameter sweeps via Python iterables, ensuring reproducibility by tracing every result to its configuration snapshot. Bayesian hyperparameter optimization via Optuna \cite{akiba2019optuna} searches algorithm-specific parameter spaces using multi-fidelity evaluation, though this functionality remains under active development as ongoing research into problem-specific optimal configurations. Complete technical documentation, installation procedures, usage workflows, and extension guidelines are provided in Appendix~\ref{att:setup_guide}.

\section{Interactive Optimal Stopping Game}
\label{sec:pricing_game}

To demonstrate optimal stopping theory beyond academic audiences and provide an accessible educational tool, an interactive web application gamifies the American option exercise decision problem. Players compete against pre-trained SRLSM algorithms by observing simulated stock price paths unfolding step-by-step and deciding whether to exercise immediately or continue. The machine opponent applies its learned optimal policy in parallel, with final payoffs compared to determine the winner.

\subsection{Game Design and Mechanics}
\label{subsec:game_mechanics}

Two game modes highlight different pricing complexities:

\begin{enumerate}
    \item \textbf{Up-and-Out Min Put (3 stocks):} A rainbow option with payoff $(\max(0, K - \min(S^1_t, S^2_t, S^3_t)))$ and upper barrier at 110, demonstrating multi-asset correlation effects and knockout risk.

    \item \textbf{Double Knock-Out Lookback Put (1 stock):} A path-dependent contract with payoff $\max_{\tau \le t} S_\tau - S_T$ and barriers at 90 and 110, illustrating memory-dependent derivatives where players track running maxima while navigating two-sided knockout corridors.
\end{enumerate}

\noindent Both employ Geometric Brownian Motion with $r = 0.02$, $\sigma = 0.2$, $S_0 = K = 100$, consistent with Chapter~\ref{ch:results} benchmarks, creating balanced risk-reward profiles where optimal exercise timing is non-trivial.

\subsection{Implementation and Interface}
\label{subsec:game_implementation}

The application combines a React frontend with a Flask backend serving pre-generated paths and model predictions via REST API. To eliminate latency, SRLSM algorithms are trained offline on 50,000 paths per mode, with 500 test paths stored for instant initialization. The interface adopts retro arcade aesthetics (pixel fonts, neon colors, CRT effects) to enhance engagement. Stock prices animate with smooth interpolation, exercise decisions receive visual feedback indicating alignment with machine policy, and real-time statistics track cumulative payoffs and decision agreement rates.

\vspace{0.5cm}

\noindent Figures~\ref{fig:game_menu} and \ref{fig:game_play} illustrate the interface. The main menu presents scenario selection with contract specifications and difficulty indicators, while gameplay displays synchronized price charts with barrier level markers, decision prompts, and comparative payoff tracking.

\begin{figure}[h]
\centering
% \includegraphics[width=0.7\textwidth]{Chapter7/figures/game_menu.pdf}
\caption{Main menu allowing selection between Up-and-Out Min Put and Double Knock-Out Lookback Put scenarios with contract specifications.}
\label{fig:game_menu}
\end{figure}

\begin{figure}[h]
\centering
% \includegraphics[width=0.85\textwidth]{Chapter7/figures/game_play.pdf}
\caption{Gameplay interface showing real-time stock price evolution, barrier levels, exercise decision prompts, and cumulative payoff comparison with color-coded decision agreement indicators.}
\label{fig:game_play}
\end{figure}

\subsection{Educational Value}
\label{subsec:game_value}

The game serves dual purposes. First, it transforms abstract mathematical objects from Chapter~\ref{ch:methods} into tangible decision-making agents, making the research accessible to finance practitioners, students, and general audiences unfamiliar with stochastic calculus. Second, it functions as a pedagogical tool where users develop intuition for time value erosion, volatility impact, and early exercise premiums through experiencing consequences of suboptimal decisions. Preliminary classroom deployment suggests students interacting with the game before studying American option theory demonstrate improved conceptual understanding and retention compared to lecture-only instruction. The complete source code with deployment configurations is available in the \texttt{thesis-game/} repository subdirectory.

\section{Reproducibility and Accessibility}
\label{sec:accessibility}

All software components prioritize transparency and replication. The GitHub repository includes comprehensive documentation, and every Chapter~\ref{ch:results} experiment is tagged with configuration metadata and git commit hashes enabling exact reproduction. Pre-generated path datasets are publicly hosted on Google Drive, eliminating computational barriers for researchers lacking high-performance resources. Detailed environment setup, dependency installation, execution workflows, convergence plot generation, and codebase extension procedures are provided in Appendix~\ref{att:setup_guide}.

\vspace{0.5cm}

\noindent By open-sourcing this production-grade implementation with extensive documentation, this research lowers barriers to future investigations into randomized neural network methods for American option pricing. The modular architecture facilitates extensions to new algorithms, exotic derivatives, and alternative stochastic models, positioning the codebase as a platform for continued methodological development within the computational finance community.
