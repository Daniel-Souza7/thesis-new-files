\chapter{How to Run the Code}
\label{appendix:code}

% ============================================================================
% REQUIRED PACKAGES (add to your main thesis preamble if not already present):
% ----------------------------------------------------------------------------
% \usepackage{listings}
% \usepackage{xcolor}
%
% LISTINGS CONFIGURATION (add to preamble):
% ----------------------------------------------------------------------------
\definecolor{codebg}{RGB}{248,248,248}
\definecolor{codeframe}{RGB}{200,200,200}
\definecolor{codekeyword}{RGB}{0,112,32}
\definecolor{codestring}{RGB}{186,33,33}
\definecolor{codecomment}{RGB}{96,96,96}
\definecolor{codenumber}{RGB}{64,128,128}

\lstset{
    basicstyle=\ttfamily\small,
    backgroundcolor=\color{codebg},
    frame=single,
    rulecolor=\color{codeframe},
    framesep=3pt,
    xleftmargin=6pt,
    xrightmargin=6pt,
    breaklines=true,
    breakatwhitespace=true,
    showstringspaces=false,
    tabsize=4,
    aboveskip=8pt,
    belowskip=4pt,
    keywordstyle=\color{codekeyword}\bfseries,
    stringstyle=\color{codestring},
    commentstyle=\color{codecomment}\itshape,
    numberstyle=\tiny\color{codenumber},
    morekeywords={python,git,pip,source,cd,from,import,dataclass,class,def,True,False,None,tuple,int,float,str,bool},
}
% ============================================================================

This appendix provides comprehensive instructions for reproducing the experimental results presented in this thesis. The \texttt{optimal\_stopping} software framework implements all algorithms, payoff structures, and stochastic process models described in the preceding chapters. Complete source code is available at:
\begin{center}
\vspace{-4pt}
\url{https://github.com/Daniel-Souza7/thesis-new-files}
\vspace{-2pt}
\end{center}

\section{Installation and Setup}
\label{sec:installation}

\subsection{System Requirements}
\vspace{-4pt}

The software requires the following minimum specifications:
\vspace{-4pt}
\begin{itemize}
    \setlength{\itemsep}{1pt}
    \item \textbf{Python:} Version 3.8 or higher
    \item \textbf{Memory:} 8 GB RAM minimum; 32+ GB recommended for high-dimensional experiments ($d \geq 50$)
    \item \textbf{Storage:} 10 GB for pre-computed path datasets
    \item \textbf{GPU:} CUDA-compatible GPU optional (for deep learning methods DOS and NLSM only)
\end{itemize}

\subsection{Installation Procedure}
\vspace{-4pt}

\begin{enumerate}
    \setlength{\itemsep}{2pt}
    \item \textbf{Clone the repository:}
    \vspace{-4pt}
    \begin{lstlisting}[language=bash]
git clone https://github.com/Daniel-Souza7/thesis-new-files.git
cd thesis-new-files
    \end{lstlisting}

    \vspace{-6pt}
    \item \textbf{Create a virtual environment (recommended):}
    \vspace{-4pt}
    \begin{lstlisting}[language=bash]
python -m venv venv
source venv/bin/activate  # Linux/macOS
venv\Scripts\activate     # Windows
    \end{lstlisting}

    \vspace{-6pt}
    \item \textbf{Install dependencies:}
    \vspace{-4pt}
    \begin{lstlisting}[language=bash]
pip install -r requirements.txt
    \end{lstlisting}

    \vspace{-6pt}
    \item \textbf{Verify installation:}
    \vspace{-4pt}
    \begin{lstlisting}[language=Python]
python -c "from optimal_stopping.payoffs import list_payoffs; \
           print(f'Loaded {len(list_payoffs())} payoffs')"
    \end{lstlisting}
    \vspace{-4pt}
    Expected output: \texttt{Loaded 360 payoffs}
\end{enumerate}

\section{Configuration System}
\label{sec:config-system}
\vspace{-4pt}

The experimental framework employs a declarative configuration system where all parameters are specified in \texttt{optimal\_stopping/run/configs.py}. Each configuration is defined as a Python dataclass with parameters specified as iterables, enabling automatic grid search over all parameter combinations.

\subsection{Configuration Structure}
\vspace{-4pt}

A minimal configuration has the following structure:
\vspace{-2pt}
\begin{lstlisting}[language=Python]
from dataclasses import dataclass

@dataclass
class my_experiment(_DefaultConfig):
    # Algorithm selection
    algos: tuple = ('RT', 'RLSM', 'LSM')

    # Problem specification
    payoffs: tuple = ('BasketCall', 'BasketPut')
    stock_models: tuple = ('BlackScholes',)
    nb_stocks: tuple = (5, 50)           # Dimension d

    # Market parameters (as in Section 3.2)
    drift: tuple = (0.05,)               # Risk-free rate r
    volatilities: tuple = (0.2,)         # Volatility sigma
    spots: tuple = (100,)                # Initial price S_0
    strikes: tuple = (100,)              # Strike price K
    maturities: tuple = (1.0,)           # Time to maturity T

    # Simulation parameters
    nb_paths: tuple = (1000000,)         # Monte Carlo paths m
    nb_dates: tuple = (100,)             # Exercise dates N
    nb_runs: int = 5                     # Repetitions

    # Algorithm hyperparameters (as in Section 3.1)
    hidden_size: tuple = (75,)           # Hidden neurons K
    activation: tuple = ('leakyrelu',)   # Activation function
    use_payoff_as_input: tuple = (True,) # Payoff augmentation
    train_ITM_only: tuple = (True,)      # ITM filtering
\end{lstlisting}

\vspace{-2pt}
The configuration above generates $3 \times 2 \times 2 = 12$ experimental combinations (3 algorithms $\times$ 2 payoffs $\times$ 2 dimensions), each repeated 5 times.

\subsection{Key Configuration Parameters}
\label{subsec:config-params}
\vspace{-4pt}

Table~\ref{tab:config-params} summarizes the principal configuration parameters, with references to their theoretical definitions in the thesis.

\vspace{-4pt}
\begin{table}[htbp]
\centering
\caption{Configuration parameters and their correspondence to thesis notation.}
\label{tab:config-params}
\vspace{-4pt}
\begin{tabular}{llll}
\toprule
\textbf{Parameter} & \textbf{Type} & \textbf{Thesis Symbol} & \textbf{Reference} \\
\midrule
\texttt{nb\_stocks} & \texttt{int} & $d$ & Sec.~2.1.1 \\
\texttt{nb\_paths} & \texttt{int} & $m$ & Sec.~4.1.2 \\
\texttt{nb\_dates} & \texttt{int} & $N$ & Sec.~2.1.1 \\
\texttt{drift} & \texttt{float} & $r$ & Eq.~(3.4) \\
\texttt{volatilities} & \texttt{float} & $\sigma$ & Eq.~(3.4) \\
\texttt{spots} & \texttt{float} & $S_0$ & Sec.~2.1.1 \\
\texttt{strikes} & \texttt{float} & $K$ & Sec.~2.1.1 \\
\texttt{maturities} & \texttt{float} & $T$ & Sec.~2.1.1 \\
\texttt{hidden\_size} & \texttt{int} & $K$ & Sec.~2.3.1 \\
\texttt{activation} & \texttt{str} & $\sigma(\cdot)$ & Sec.~2.3.1 \\
\texttt{barriers} & \texttt{float} & $B$ & Sec.~3.3.2 \\
\texttt{barriers\_up} & \texttt{float} & $B_U$ & Sec.~3.3.2 \\
\texttt{barriers\_down} & \texttt{float} & $B_L$ & Sec.~3.3.2 \\
\texttt{hurst} & \texttt{float} & $H$ & Sec.~3.2.3 \\
\texttt{mean} & \texttt{float} & $\theta$ & Eq.~(3.8) \\
\texttt{speed} & \texttt{float} & $\kappa$ & Eq.~(3.8) \\
\bottomrule
\end{tabular}
\end{table}

\section{Running Experiments}
\label{sec:running}

\subsection{Basic Execution}
\vspace{-4pt}

Once a configuration is defined in \texttt{configs.py}, experiments are executed via the command line:
\vspace{-2pt}
\begin{lstlisting}[language=bash]
python -m optimal_stopping.run.run_algo --configs=my_experiment
\end{lstlisting}

\vspace{-4pt}
Results are automatically saved to \texttt{optimal\_stopping/run/results/my\_experiment/}.

\subsection{Results Aggregation}
\vspace{-4pt}

To aggregate results into an Excel file with summary statistics:
\vspace{-2pt}
\begin{lstlisting}[language=bash]
python -m optimal_stopping.run.write_excel --configs=my_experiment
\end{lstlisting}

\vspace{-4pt}
This generates \texttt{my\_experiment\_summary.xlsx} containing mean prices, standard deviations, and execution times for each configuration.

\subsection{Command-Line Options}
\vspace{-4pt}

The execution script supports several options for filtering and customization:
\vspace{-2pt}
\begin{lstlisting}[language=bash]
python -m optimal_stopping.run.run_algo --configs=my_experiment \
    --nb_jobs=8 \              # Parallel CPU workers
    --algos=RT,RLSM \          # Filter algorithms
    --nb_stocks=50 \           # Filter dimensions
    --path_gen_seed=42 \       # Reproducible paths
    --print_errors             # Show full error traces
\end{lstlisting}

\section{Reproducing Thesis Results}
\label{sec:reproducing}
\vspace{-4pt}

This section provides the exact configurations needed to reproduce the principal experimental results presented in Chapter~4.

\subsection{Table 4.2: Dimensional Scalability Benchmark}
\vspace{-4pt}

To reproduce the cross-algorithm benchmark across dimensions $d \in \{1, 2, 7, 50, 500\}$:
\vspace{-2pt}
\begin{lstlisting}[language=Python]
@dataclass
class table_4_2_dimensional_scalability(_DefaultConfig):
    """Reproduce Table 4.2: Algorithmic comparison."""
    algos: tuple = ('RT', 'RLSM', 'LSM', 'DOS', 'NLSM', 'EOP')
    payoffs: tuple = ('BasketCall',)
    stock_models: tuple = ('BlackScholes',)

    nb_stocks: tuple = (1, 2, 7, 50, 500)
    nb_paths: tuple = (10000000,)
    nb_dates: tuple = (100,)
    nb_runs: int = 5

    drift: tuple = (0.08,)
    volatilities: tuple = (0.2,)
    dividends: tuple = (0.0,)
    spots: tuple = (100,)
    strikes: tuple = (100,)
    maturities: tuple = (1.0,)

    hidden_size: tuple = (75,)
    activation: tuple = ('leakyrelu',)
    use_payoff_as_input: tuple = (True,)
    dtype: tuple = ('float32',)
\end{lstlisting}

\vspace{-2pt}
Execute with:
\vspace{-2pt}
\begin{lstlisting}[language=bash]
python -m optimal_stopping.run.run_algo \
    --configs=table_4_2_dimensional_scalability
python -m optimal_stopping.run.write_excel \
    --configs=table_4_2_dimensional_scalability
\end{lstlisting}

\subsection{Table 4.3: MaxCall Activation Function Validation}
\vspace{-4pt}

To reproduce the activation function comparison on MaxCall options:
\vspace{-2pt}
\begin{lstlisting}[language=Python]
@dataclass
class table_4_3_maxcall_activation(_DefaultConfig):
    """Reproduce Table 4.3: MaxCall RT vs RLSM."""
    algos: tuple = ('RT', 'RLSM')
    payoffs: tuple = ('MaxCall',)
    stock_models: tuple = ('BlackScholes',)

    nb_stocks: tuple = (5, 25, 250)
    nb_paths: tuple = (5000000,)
    nb_dates: tuple = (100,)
    nb_runs: int = 5

    drift: tuple = (0.08,)
    volatilities: tuple = (0.2,)
    spots: tuple = (100,)
    strikes: tuple = (100,)
    maturities: tuple = (1.0,)

    # RT uses adaptive activation (ELU for MaxCall)
    activation: tuple = ('elu', 'leakyrelu')
    hidden_size: tuple = (50,)
\end{lstlisting}

\subsection{Table 4.8: Path-Dependent Performance (RT vs RRLSM)}
\vspace{-4pt}

To reproduce the path-dependent option comparison:
\vspace{-2pt}
\begin{lstlisting}[language=Python]
@dataclass
class table_4_8_path_dependent(_DefaultConfig):
    """Reproduce Table 4.8: RT vs RRLSM."""
    algos: tuple = ('RT', 'RRLSM')
    payoffs: tuple = (
        'LookbackFixedCall',
        'LookbackFloatPut',
        'AsianFixedStrikePut',
        'AsianFloatingStrikeCall',
        'UO-DispersionCall',
        'DO-BestOfKCall',
        'UI-MinPut',
        'DI-MaxCall'
    )
    stock_models: tuple = ('BlackScholes',)

    nb_stocks: tuple = (1, 5, 25, 50, 250, 500)
    nb_paths: tuple = (2000000,)
    nb_dates: tuple = (50,)

    # Barrier parameters
    barriers: tuple = (150,)
    barriers_up: tuple = (150,)
    barriers_down: tuple = (70,)

    # Rank parameters for BestOfK
    k: tuple = (2,)
    weights: tuple = ((0.5, 0.5),)

    activation: tuple = ('elu',)
\end{lstlisting}

\subsection{Table 4.5: Barrier Option Convergence}
\vspace{-4pt}

To validate barrier option convergence under extreme thresholds:
\vspace{-2pt}
\begin{lstlisting}[language=Python]
@dataclass
class table_4_5_barrier_convergence(_DefaultConfig):
    """Reproduce Table 4.5: Barrier convergence."""
    algos: tuple = ('RT',)
    payoffs: tuple = ('UO-BasketCall', 'DO-MaxCall')
    stock_models: tuple = ('BlackScholes',)

    nb_stocks: tuple = (5, 25)
    nb_paths: tuple = (5000000,)

    # Extreme barriers for convergence validation
    barriers: tuple = (1000, 0.001)  # Far OTM barriers

    spots: tuple = (90, 100, 110)  # OTM, ATM, ITM
    strikes: tuple = (100,)
\end{lstlisting}

\section{Configuring Barrier Options}
\label{sec:barrier-config}
\vspace{-4pt}

Barrier options are specified using a prefix notation in the payoff name. The naming convention follows the pattern \texttt{\{BARRIER\}-\{BASE\_PAYOFF\}}.

\subsection{Single Barrier Configuration}
\vspace{-4pt}
\begin{lstlisting}[language=Python]
@dataclass
class single_barrier_example(_DefaultConfig):
    algos: tuple = ('RT', 'SRLSM')
    payoffs: tuple = (
        'UO-BasketCall',    # Up-and-Out
        'DO-MaxCall',       # Down-and-Out
        'UI-MinPut',        # Up-and-In
        'DI-GeometricPut'   # Down-and-In
    )

    barriers: tuple = (120,)  # Single barrier level B
    spots: tuple = (100,)
    strikes: tuple = (100,)
\end{lstlisting}

\subsection{Double Barrier Configuration}
\vspace{-4pt}
\begin{lstlisting}[language=Python]
@dataclass
class double_barrier_example(_DefaultConfig):
    algos: tuple = ('RT',)
    payoffs: tuple = (
        'UODO-BasketCall',  # Double Knock-Out
        'UIDI-MinPut',      # Double Knock-In
        'UIDO-MaxCall',     # Up-In-Down-Out
        'UODI-GeometricCall' # Up-Out-Down-In
    )

    barriers_up: tuple = (150,)    # Upper barrier B_U
    barriers_down: tuple = (70,)   # Lower barrier B_L
\end{lstlisting}

\subsection{Step Barrier Configuration}
\vspace{-4pt}
\begin{lstlisting}[language=Python]
@dataclass
class step_barrier_example(_DefaultConfig):
    algos: tuple = ('RT',)
    payoffs: tuple = (
        'StepB-BasketCall',   # Single step barrier
        'DStepB-MinPut'       # Double step barrier
    )

    # Step barrier parameters
    step_param1: tuple = (100,)  # B(0) start level
    step_param2: tuple = (150,)  # B(T) end level
    step_param3: tuple = (80,)   # Lower B(0) for double
    step_param4: tuple = (60,)   # Lower B(T) for double
\end{lstlisting}

\section{Configuring Stochastic Processes}
\label{sec:process-config}
\vspace{-4pt}

The framework supports five stochastic process models as described in Section~3.2.

\subsection{Black-Scholes (GBM)}
\vspace{-4pt}
\begin{lstlisting}[language=Python]
@dataclass
class blackscholes_config(_DefaultConfig):
    stock_models: tuple = ('BlackScholes',)

    drift: tuple = (0.05,)        # r
    volatilities: tuple = (0.2,)  # sigma
    correlation: tuple = (0.3,)   # rho (cross-asset)
    dividends: tuple = (0.0,)     # q
\end{lstlisting}

\subsection{Heston Stochastic Volatility}
\vspace{-4pt}
\begin{lstlisting}[language=Python]
@dataclass
class heston_config(_DefaultConfig):
    stock_models: tuple = ('Heston',)

    drift: tuple = (0.05,)        # r
    volatilities: tuple = (0.2,)  # sqrt(v_0)
    mean: tuple = (0.04,)         # theta (long-run variance)
    speed: tuple = (2.0,)         # kappa (mean reversion)
    correlation: tuple = (-0.7,)  # rho (leverage effect)
\end{lstlisting}

\subsection{Fractional Brownian Motion}
\vspace{-4pt}
\begin{lstlisting}[language=Python]
@dataclass
class fbm_config(_DefaultConfig):
    stock_models: tuple = ('FractionalBrownianMotion',)

    hurst: tuple = (0.3, 0.5, 0.7)  # H parameter
    drift: tuple = (0.05,)
    volatilities: tuple = (0.2,)
\end{lstlisting}

\subsection{Rough Heston}
\vspace{-4pt}
\begin{lstlisting}[language=Python]
@dataclass
class rough_heston_config(_DefaultConfig):
    stock_models: tuple = ('RoughHeston',)

    hurst: tuple = (0.1,)          # H (roughness)
    mean: tuple = (0.3,)           # theta
    speed: tuple = (0.15,)         # kappa
    correlation: tuple = (-0.7,)   # rho
\end{lstlisting}

\subsection{Stationary Block Bootstrap (Real Data)}
\vspace{-4pt}
\begin{lstlisting}[language=Python]
@dataclass
class real_data_config(_DefaultConfig):
    stock_models: tuple = ('RealData',)

    nb_stocks: tuple = (5, 10, 25)
    nb_paths: tuple = (500000,)
    maturities: tuple = (0.5,)
    # Uses S&P 500 historical data via Yahoo Finance
\end{lstlisting}

\section{Algorithm-Specific Configuration}
\label{sec:algo-config}

\subsection{RT Algorithm (Proposed)}
\vspace{-4pt}

The RT algorithm (Section~3.1) accepts the following hyperparameters:
\vspace{-2pt}
\begin{lstlisting}[language=Python]
hidden_size: tuple = (75,)           # K neurons
activation: tuple = ('leakyrelu',)   # sigma(.)
use_payoff_as_input: tuple = (True,) # Payoff augmentation
train_ITM_only: tuple = (True,)      # ITM filtering
dropout: tuple = (0.0,)              # Dropout probability
\end{lstlisting}

\vspace{-2pt}
The dimension-adaptive heuristic from Equation~(3.1) suggests:
\vspace{-4pt}
\begin{itemize}
    \setlength{\itemsep}{1pt}
    \item $d \leq 9$: $K = \max(2d, 5)$
    \item $10 \leq d \leq 49$: $K = 1.5d$
    \item $d \geq 500$: $K = 1.2d$
\end{itemize}

\vspace{-2pt}
For path-dependent or non-smooth payoffs (MaxCall, MinPut, barriers), use \texttt{activation='elu'}.

\subsection{RLSM Algorithm (Baseline)}
\vspace{-4pt}
\begin{lstlisting}[language=Python]
hidden_size: tuple = (20,)           # Original fixed size
activation: tuple = ('leakyrelu',)
ridge_coeff: tuple = (0.0,)          # L2 regularization
\end{lstlisting}

\subsection{LSM Algorithm (Classical)}
\vspace{-4pt}

The classical LSM algorithm (Section~2.2) uses polynomial basis functions and does not require neural network hyperparameters.

\subsection{Deep Learning Algorithms (DOS, NLSM)}
\vspace{-4pt}
\begin{lstlisting}[language=Python]
hidden_size: tuple = (40,)
nb_epochs: tuple = (30,)             # Training epochs
\end{lstlisting}

\section{Output and Results}
\label{sec:output}

\subsection{Output Directory Structure}
\vspace{-4pt}

Results are organized as follows:
\vspace{-2pt}
\begin{lstlisting}
optimal_stopping/run/results/
  my_experiment/
    my_experiment.csv           # Raw results
    my_experiment_summary.xlsx  # Aggregated statistics
    figures/                    # Generated plots
\end{lstlisting}

\subsection{CSV Output Format}
\vspace{-4pt}

The raw CSV file contains one row per experimental run with columns:

\vspace{-4pt}
\begin{table}[htbp]
\centering
\caption{CSV output columns.}
\vspace{-4pt}
\begin{tabular}{ll}
\toprule
\textbf{Column} & \textbf{Description} \\
\midrule
\texttt{algo} & Algorithm name \\
\texttt{payoff} & Payoff structure \\
\texttt{nb\_stocks} & Dimension $d$ \\
\texttt{nb\_paths} & Monte Carlo paths $m$ \\
\texttt{price} & Computed option price $\hat{p}_0$ \\
\texttt{time} & Execution time (seconds) \\
\texttt{run\_id} & Run number \\
\bottomrule
\end{tabular}
\end{table}

\section{Hyperparameter Optimization}
\label{sec:hpo}
\vspace{-4pt}

For automated hyperparameter search (Appendix~B), configure as follows:
\vspace{-2pt}
\begin{lstlisting}[language=Python]
@dataclass
class hpo_experiment(_DefaultConfig):
    algos: tuple = ('RT',)
    payoffs: tuple = ('BasketCall',)

    enable_hyperopt: bool = True
    hyperopt_method: str = 'tpe'        # Bayesian optimization
    hyperopt_timeout: float = 3600      # 1 hour
    hyperopt_n_trials: int = 100
    hyperopt_fidelity_factor: int = 4   # Use m/4 paths
    hyperopt_variance_penalty: float = 0.1
\end{lstlisting}

\vspace{-2pt}
Execute with:
\vspace{-2pt}
\begin{lstlisting}[language=bash]
python -m optimal_stopping.run.run_hyperopt --configs=hpo_experiment
\end{lstlisting}

\section{Pre-Computed Path Datasets}
\label{sec:datasets}
\vspace{-4pt}

To ensure exact reproducibility, pre-computed path datasets used in Chapter~4 are available at:
\begin{center}
\vspace{-4pt}
\url{https://drive.google.com/drive/folders/thesis-datasets}
\vspace{-2pt}
\end{center}

\subsection{Available Datasets}
\vspace{-4pt}

\begin{table}[htbp]
\centering
\caption{Pre-computed path datasets.}
\vspace{-4pt}
\begin{tabular}{lcccc}
\toprule
\textbf{ID} & \textbf{Model} & $\mathbf{d}$ & $\mathbf{m}$ & \textbf{Parameters} \\
\midrule
\texttt{BS\_1.h5} & BlackScholes & 1 & 8M & $r=0.08$, $\sigma=0.2$ \\
\texttt{BS\_2.h5} & BlackScholes & 2 & 8M & $r=0.08$, $\sigma=0.2$ \\
\texttt{BS\_7.h5} & BlackScholes & 7 & 14M & $r=0.08$, $\sigma=0.2$ \\
\texttt{BS\_50.h5} & BlackScholes & 50 & 10M & $r=0.08$, $\sigma=0.2$ \\
\texttt{BS\_500.h5} & BlackScholes & 500 & 10M & $r=0.08$, $\sigma=0.2$ \\
\texttt{RH\_5.h5} & RoughHeston & 5 & 10M & $H=0.75$, $\kappa=0.15$ \\
\texttt{SBB\_25.h5} & RealData & 25 & 10M & Historical S\&P 500 \\
\bottomrule
\end{tabular}
\end{table}

\subsection{Using Stored Paths}
\vspace{-4pt}

Download the datasets and place them in \texttt{optimal\_stopping/data/stored\_paths/}. Configure experiments to use stored paths:
\vspace{-2pt}
\begin{lstlisting}[language=Python]
@dataclass
class stored_paths_experiment(_DefaultConfig):
    stock_models: tuple = ('StoredPaths',)
    stored_paths_id: str = 'BS_50'  # Use BS_50.h5
\end{lstlisting}

\section{Troubleshooting}
\label{sec:troubleshooting}

\subsection{Memory Issues}
\vspace{-4pt}

For high-dimensional problems ($d \geq 100$) or large path counts ($m \geq 10^7$):
\vspace{-2pt}
\begin{lstlisting}[language=Python]
# Use single precision
dtype: tuple = ('float32',)

# Reduce path count
nb_paths: tuple = (1000000,)

# Reduce parallelism
python -m optimal_stopping.run.run_algo --configs=exp --nb_jobs=4
\end{lstlisting}

\subsection{Algorithm Not Found}
\vspace{-4pt}

Ensure algorithm names are spelled correctly (case-sensitive):
\vspace{-2pt}
\begin{lstlisting}[language=Python]
algos: tuple = ('RT', 'RLSM', 'LSM')  # Correct
algos: tuple = ('rt', 'rlsm', 'lsm')  # Incorrect
\end{lstlisting}

\subsection{Reproducibility}
\vspace{-4pt}

For exact reproducibility across runs:
\vspace{-2pt}
\begin{lstlisting}[language=bash]
python -m optimal_stopping.run.run_algo --configs=exp --path_gen_seed=42
\end{lstlisting}

\section{Summary}
\label{sec:summary}
\vspace{-4pt}

The \texttt{optimal\_stopping} framework provides a complete infrastructure for American option pricing research. Key usage patterns:
\vspace{-4pt}
\begin{enumerate}
    \setlength{\itemsep}{1pt}
    \item Define configuration in \texttt{configs.py}
    \item Execute: \texttt{python -m optimal\_stopping.run.run\_algo --configs=name}
    \item Aggregate: \texttt{python -m optimal\_stopping.run.write\_excel --configs=name}
    \item Analyze results in \texttt{results/name/}
\end{enumerate}

\vspace{-2pt}
For questions or issues, consult the repository documentation or contact the author.
